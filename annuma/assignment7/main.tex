\documentclass[a4paper]{article}
%\usepackage[singlespacing]{setspace}
\usepackage[onehalfspacing]{setspace}
%\usepackage[doublespacing]{setspace}
\usepackage{geometry} % Required for adjusting page dimensions and margins
\usepackage{amsmath,amsfonts,stmaryrd,amssymb,mathtools,dsfont} % Math packages
\usepackage{tabularx}
\usepackage{colortbl}
\usepackage{listings}
\usepackage{amsmath}
\usepackage{amssymb}
\usepackage{amsthm}
\usepackage{enumerate}
\usepackage{enumitem}
\usepackage{subcaption}
\usepackage{float}
\usepackage[table,xcdraw]{xcolor}
\usepackage{tikz-qtree}
\usepackage{tikz}
\usepackage{pgfplots}
\pgfplotsset{compat=1.17}
\usepackage{forest}
\usepackage{changepage,titlesec,fancyhdr} % For styling Header and Titles
\pagestyle{fancy}
\renewcommand{\headrulewidth}{0.5pt} % Linienbreite anpassen, falls gewünscht
\renewcommand{\headrule}{
    \makebox[\textwidth]{\rule{1.0\textwidth}{0.5pt}} 
}
\usepackage{amsmath}
\pagestyle{fancy}
\usepackage{diagbox}
\usepackage{xfrac}

\usepackage{enumerate} % Custom item numbers for enumerations

\usepackage[ruled]{algorithm2e} % Algorithms

\usepackage[framemethod=tikz]{mdframed} % Allows defining custom boxed/framed environments

\usepackage{listings} % File listings, with syntax highlighting
\lstset{
	basicstyle=\ttfamily, % Typeset listings in monospace font
}

\usepackage[ddmmyyyy]{datetime}

\geometry{
	paper=a4paper, % Paper size, change to letterpaper for US letter size
	top=3cm, % Top margin
	bottom=3cm, % Bottom margin
	left=2.5cm, % Left margin
	right=2.5cm, % Right margin
	headheight=25pt, % Header height
	footskip=1.5cm, % Space from the bottom margin to the baseline of the footer
	headsep=1cm, % Space from the top margin to the baseline of the header
	%showframe, % Uncomment to show how the type block is set on the page
}
\lhead{Analysis und Numerik\\Sommersemester 2025}
\chead{\bfseries{\vspace{0.5\baselineskip}Übungsblatt 7}}
\rhead{Lienkamp, 8128180\\Werner, 7987847}
\fancyheadoffset[R]{0cm}

\begin{document}
\setcounter{section}{7}
\subsection{Multiple Choice}
Kreuzen Sie bei den folgenden Fragen jeweils die zutreffende Antwort an. Korrekte Kreuze bringen +0.5 Punkte. Falsche Kreuze -0.5 Punkte. Sie bekommen auf diese Aufgabe mindestens 0 Punkte.
\begin{enumerate}[label=(\alph*)]
    \item Sei $f: \mathbb{R} \rightarrow \mathbb{R}, f(x) = x^n + x^m$ für  $n, m \in \mathbb{N}_0$. Dann ist
    \[f \in O\left(x^{max\{n, m\}}\right) \quad \text{für } x \rightarrow \infty\]
    \begin{flushright}
        wahr $\boxtimes\quad$ falsch $\square$
    \end{flushright}
    \textcolor{gray}{Bei Addition zweier Terme, kann der Term kleinerer Laufzeitklasse vernachlässigt werden. Somit zählt nur der größere. die Funktion $max\{n, m\}$ wählt den größeren der beiden Werte und somit die größere der beiden Laufzeitklassen}

    \item Die Funktion $f: \mathbb{R} \rightarrow \mathbb{R}, f(x) = x^3 - x$ ist injektiv
    \begin{flushright}
        wahr $\square\quad$ falsch $\boxtimes$
    \end{flushright}
    \textcolor{gray}{Die Injektivität ist erfüllt, wenn eine Funktion nie für zwei oder mehr Eingabewerte den selben Funktionswert liefert. $x^3 - x$ erfüllt diese Aussage allerdings nicht. (Im grob abgeschätzten Interval $-2 \leq x \leq 2$ gibt es doppelte Funktionswerte)}

    \item Die Funktion $f: \mathbb{R} \rightarrow \mathbb{R}, f(x) = x^3 - x$ ist surjektiv
    \begin{flushright}
        wahr $\boxtimes\quad$ falsch $\square$
    \end{flushright}
    \textcolor{gray}{Die Surjektivität ist erfüllt, wenn eine Funktion alle Werte in $\mathbb{R}$ erreichen kann}

    \item Die Funktion $f: \mathbb{R} \rightarrow \mathbb{R},$
    \[f(x) = \begin{cases}
        \frac{1}{x}, & x \neq 0,\\
        0, & x = 0
    \end{cases}\]
    ist bijektiv
    \begin{flushright}
        wahr $\boxtimes$ falsch $\square\quad$
    \end{flushright}
    \textcolor{gray}{Die Funktion ist Surjektiv, da der Fall $\frac{1}{x}$ das Bild $\mathbb{R}/\{0\}$ abdeckt und der Fall $0$ die Bildmenge $\{0\}$. Somit gilt $\mathbb{R}/\{0\} \,\cap \,\{0\} = \mathbb{R}$. Die Bijektivität ist gegeben, da keine 2 Eingabewerte den gleichen Funktionswert erreichen.}
\end{enumerate}
\clearpage
\subsection{Rechenaufgabe}
Für die folgenden Funktionen $f: \mathbb{R} \rightarrow \mathbb{R}$ bestimmen Sie das minimale $n \in \mathbb{N}_0$, sodass
\[f \in O\left(x^n\right) \quad \text{für} \rightarrow \infty\]
Falls $f \notin O\left(x^n\right)$ für alle $n \in N_0$, tragen Sie NaN in das entrsprechende Kästchen. Bitte geben SIe nur die Endergebnisse an. Korrekte Lösungen bringen +0.5 Punkte, falsche Lösungen 0 Punkte.\\
\(f(x) = \sum\limits^7_{k=1} \frac{x^k}{k!}\)\\\\
\fbox{\parbox{\linewidth}{
\[f \in O(x^n) \quad \text{für} \, n = 7\]
}}\vspace*{2mm}
\textcolor{gray}{Die höchste Potenz ist $x^7$, also ist das asymptotische Verhalten $O(x^7)$.}\\

\(f(x) = e^x\)\\\\
\fbox{\parbox{\linewidth}{
\[f \in O(x^n) \quad \text{für} \, n = \text{NaN}\]
}}\vspace*{2mm}
\textcolor{gray}{$e^x$ wächst schneller als jede Potenz, also ist $f \notin O(x^n)$ für endliches $n$.}\\

\(f(x) = \text{sin}(x)\)\\\\
\fbox{\parbox{\linewidth}{
\[f \in O(x^n) \quad \text{für} \, n = 0\]
}}\vspace*{2mm}
\textcolor{gray}{$\sin(x)$ ist beschränkt, also ist $f \in O(1) = O(x^0)$.}\\

\(f(x) = (2 \, \text{ln} \, x - \text{ln}(x^2 + 2))\)\\\\
\fbox{\parbox{\linewidth}{
\[f \in O(x^n) \quad \text{für} \, n = 0\]
}}\vspace*{2mm}
\textcolor{gray}{Für große $x$ ist $f(x) \approx 2\ln x - 2\ln x = 0$, also beschränkt $\Rightarrow O(1) = O(x^0)$.}
\clearpage
\subsection{Schriftliche Aufgabe}
\begin{enumerate}
    \item $f: [0, 1] \rightarrow [0, 1]$ eine stetige Funktion. Zeigen Sie, dass $f$ einen Fixpunkt besitzt, d.h., es gibt ein $a \in [0, 1]$ mit $f(a) = a$.

    Wir definieren $g(x) := f(x) - x$. $g$ ist stetig, da $f$ stetig ist.
    \begin{itemize}
        \item $g(0) = f(0) - 0 \geq 0$
        \item $g(1) = f(1) - 1 \leq 0$
    \end{itemize}
    Aus der Vorlesung kennen wir den Zwischenwertsatz, welcher besagt, dass $a \in [0,1]$ mit $g(a) = 0 \Rightarrow f(a) = a$ existiert.

    \item Geben Sie ein Beispiel einer stetigen Funktion $f: (0, 1) \rightarrow (0,1 )$ an, die keinen Fixpunkt besitzt.

    Beispiel: $f(x) = \frac{1}{2} + \frac{x}{2}$  
    Diese Funktion ist auf $(0,1)$ stetig und für alle $x$ gilt: $f(x) > x$.  
    Daher existiert kein $x$ mit $f(x) = x$.
\end{enumerate}

\subsection{Votieraufgabe}
Sei $D \subseteq \mathbb{R}, f: D \rightarrow$. Beweisen oder wiederlegen Sie
\begin{enumerate}[label=(\alph*)]
\item Ist $f$ Lipschitz-stetig, so ist $f$ auch gleichmäßig stetig.

\textbf{Wahr}.\\  
Lipschitz-Stetigkeit impliziert gleichmäßige Stetigkeit, da $|f(x) - f(y)| \leq L |x - y|$ für alle $x,y \Rightarrow \varepsilon$-$\delta$-Kriterium gleichmäßig erfüllt.

\item Ist $f$ gleichmäßig stetig, so ist $f$ auch Lipschitz-stetig.
 
\textbf{Falsch.}\\  
\textit{Gegenbeispiel}: $f(x) = \sqrt{x}$ auf $[0,1]$ ist gleichmäßig stetig, aber nicht Lipschitz-stetig (Ableitung wird unbeschränkt nahe $x = 0$).

\item Ist $f$ stetig, so ist $f$ auch gleichmäßig stetig.

\textbf{Falsch.}\\
\textit{Gegenbeispiel}: $f(x) = \tan(x)$ auf $(-\frac{\pi}{2}, \frac{\pi}{2})$ ist stetig, aber nicht gleichmäßig stetig wegen Unbeschränktheit der Ableitung nahe den Randpunkten.
\end{enumerate}
\end{document}
