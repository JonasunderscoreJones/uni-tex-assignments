\documentclass[a4paper]{article}
%\usepackage[singlespacing]{setspace}
\usepackage[onehalfspacing]{setspace}
%\usepackage[doublespacing]{setspace}
\usepackage{geometry} % Required for adjusting page dimensions and margins
\usepackage{amsmath,amsfonts,stmaryrd,amssymb,mathtools,dsfont} % Math packages
\usepackage{tabularx}
\usepackage{colortbl}
\usepackage{listings}
\usepackage{amsmath}
\usepackage{amssymb}
\usepackage{amsthm}
\usepackage{enumerate}
\usepackage{enumitem}
\usepackage{subcaption}
\usepackage{float}
\usepackage[table,xcdraw]{xcolor}
\usepackage{tikz-qtree}
\usepackage{forest}
\usepackage{changepage,titlesec,fancyhdr} % For styling Header and Titles
\pagestyle{fancy}
\renewcommand{\headrulewidth}{0.5pt} % Linienbreite anpassen, falls gewünscht
\renewcommand{\headrule}{
    \makebox[\textwidth]{\rule{1.0\textwidth}{0.5pt}} 
}
\usepackage{amsmath}
\pagestyle{fancy}
\usepackage{diagbox}
\usepackage{xfrac}

\usepackage{enumerate} % Custom item numbers for enumerations

\usepackage[ruled]{algorithm2e} % Algorithms

\usepackage[framemethod=tikz]{mdframed} % Allows defining custom boxed/framed environments

\usepackage{listings} % File listings, with syntax highlighting
\lstset{
	basicstyle=\ttfamily, % Typeset listings in monospace font
}

\usepackage[ddmmyyyy]{datetime}


\geometry{
	paper=a4paper, % Paper size, change to letterpaper for US letter size
	top=3cm, % Top margin
	bottom=3cm, % Bottom margin
	left=2.5cm, % Left margin
	right=2.5cm, % Right margin
	headheight=25pt, % Header height
	footskip=1.5cm, % Space from the bottom margin to the baseline of the footer
	headsep=1cm, % Space from the top margin to the baseline of the header
	%showframe, % Uncomment to show how the type block is set on the page
}
\lhead{Analysis und Numerik\\Sommersemester 2025}
\chead{\bfseries{\vspace{0.5\baselineskip}Übungsblatt 3}}
\rhead{Lienkamp, 8128180\\Werner, 7987847}
\fancyheadoffset[R]{0cm}

\begin{document}
\setcounter{section}{3}
\subsection{Multiple Choice}
Kreuzen Sie bei den folgenden Fragen jeweils die zutreffende Antwort an. Korrekte Kreuze bringen +0.5 Punkte. Falsche Kreuze -0.5 Punkte. Sie bekommen auf diese Aufgabe mindestens 0 Punkte.

\begin{enumerate}[label=(\alph*)]
    \item Jede konvergente reelle Folge ist bereits eine Cauchy-Folge. \begin{flushright} wahr $\square \quad$ falsch $\boxtimes$ \end{flushright}
    \item Seien $a \in \mathbb{R}$ und $(a_n)_n \in \mathbb{N} \subset \mathbb{R}$ eine reelle Folge. Seien des Weiteren die Teilfolgen $(a_{3n})_n \in \mathbb{N}$, $(a_{3n+1})_n \in \mathbb{N}$, $(a_{3n+2})_n \in \mathbb{N} \subset \mathbb{R}$ konvergent mit
    \[
    \lim\limits_{n \to \infty} a_{3n} = \lim\limits_{n \to \infty} a_{3n + 1} = \lim\limits_{n \to \infty} a_{3n + 2} = a
    \]
    Dann konvergiert $(a_n)_n \in \mathbb{N}$ gegen a.
    \begin{flushright} wahr $\boxtimes \quad$ falsch $\square$ \end{flushright}
    \item Seien $(a_n)_n \in \mathbb{N}$, $(b_n)_n \in N \subset \mathbb{R}$ reelle Folgen mit $c_n := a_n + b_n$ für alle $n \in \mathbb{N}$. Dann gilt
    \[
        sup\{c_n : n \in \mathbb{N}\} = sup\{a_n : n \in \mathbb{N}\} + sup\{b_n : n \in \mathbb{N}\}.
    \]
    \begin{flushright} wahr $\square \quad$ falsch $\boxtimes$ \end{flushright}
    \item Seien $(a_n)_n \in \mathbb{N}$, $(b_n)_n \in N \subset \mathbb{R}$ reelle Folgen mit $c_n := a_n - b_n$ für alle $n \in \mathbb{N}$. Dann gilt
    \[
        sup\{c_n : n \in \mathbb{N}\} \leq sup\{a_n : n \in \mathbb{N}\} - inf\{b_n : n \in \mathbb{N}\}.
    \]
    \begin{flushright} wahr $\boxtimes \quad$ falsch $\boxtimes$ \end{flushright}
\end{enumerate}
\bigskip
\subsection{Schriftliche Aufgabe}
\begin{enumerate}[label=(\alph*)]
    \item Zeigen Sie durch Verwendung von Defnition und Satz 1.19, dass die Folge
    \[
        a_{n+1} = \frac{1}{2} \left(a_n + \frac{2}{a_n} \right) \text{ mit } a_1 = 1
    \]
    konvergiert und berechnen Sie dann mit Hilfe der Rechenregeln für Grenzwerte $\lim\limits_{n \to \infty}$.\\\\
    Laut dem Satz 1.19 aus dem Skript gilt $a_{n + 1} \leq a_n$ und wenn die Folge ins positive konvergiert, ist sie begrenzt:\\
    \[
        a_{n + 1} \leq a_n
    \]
    \[
        \frac{1}{2}\left(a_n + \frac{2}{a_n}\right) \leq a_n
    \]
    \[
        a_n + \frac{2}{a_n} \leq 2a_n 
    \]
    \[
        \frac{2}{a_n} \leq a_n
    \]
    \[
        2 \leq a_n^2
    \]
    \[
        \sqrt{2} \leq a_n
    \]
Somit ist klar, dass die Folge gegen $\sqrt{2}$ konvergiert für $n \to \infty$ und von $> \sqrt{2}$ nach $\sqrt{2}$. Die Folge ist daher monoton fallend.
    \item Auf Aufgabenblatt 1 haben Sie gezeigt, dass $\sqrt{2}$ eine irrationale Zahl ist, ohne ihre Existenz zu beweisen. Zeigen Sie, dass es eine reelle Zahl $x \in \mathbb{R}$ gibt mit $x^2 = 2$.\\\\
    Die Folge $a_n$ konvergiert gegen $\sqrt{2}$ (also für $n = \infty \to a_n = \sqrt{2}$) und daher existiert ein $x \in \mathbb{R}$ mit $x^2 = 2 \Rightarrow x = \sqrt{2}$.
\end{enumerate}
\bigskip
\subsection{Rechenaufgabe}
Bestimmen Sie für die Menge
\[
    M =  \left\{ \frac{1}{m} + \frac{1}{3^n} : m, n \in \mathbb{N} \right\}
\]
das Supremum, Infmum, Maximum und Minimum, sofern jeweils existierend. Tragen Sie das Endergebnis in die Kästen ein. Im Falle der Nicht-Existenz schreiben Sie NaN in das entsprechende Kästchen. Bitte geben Sie nur die Endergebnisse an.\\
\begin{itemize}
    \item $sup M = \frac{4}{3}$
    \item $max M = \frac{4}{3}$
    \item $inf M = 0$
    \item $min M = NaN$
\end{itemize}
\bigskip
\subsection{Votieraufgabe}
Es seien A, B nicht leere, beschränkte Teilmengen von R und $A + B := \{a + b : a \in A, b \in B\}$ ihre (elementweise) Summe. Zeigen Sie: A + B ist beschränkt, und es gilt
\[
    sup(A + B) = sup A + sup B,
\]
\[
    inf(A + B) = inf A + inf B.
\]

\end{document}