\documentclass[a4paper]{article}
%\usepackage[singlespacing]{setspace}
\usepackage[onehalfspacing]{setspace}
%\usepackage[doublespacing]{setspace}
\usepackage{geometry} % Required for adjusting page dimensions and margins
\usepackage{amsmath,amsfonts,stmaryrd,amssymb,mathtools,dsfont} % Math packages
\usepackage{tabularx}
\usepackage{colortbl}
\usepackage{listings}
\usepackage{amsmath}
\usepackage{amssymb}
\usepackage{amsthm}
\usepackage{enumerate}
\usepackage{enumitem}
\usepackage{subcaption}
\usepackage{float}
\usepackage[table,xcdraw]{xcolor}
\usepackage{tikz-qtree}
\usepackage{forest}
\usepackage{changepage,titlesec,fancyhdr} % For styling Header and Titles
\pagestyle{fancy}
\renewcommand{\headrulewidth}{0.5pt} % Linienbreite anpassen, falls gewünscht
\renewcommand{\headrule}{
    \makebox[\textwidth]{\rule{1.0\textwidth}{0.5pt}} 
}
\usepackage{amsmath}
\pagestyle{fancy}
\usepackage{diagbox}
\usepackage{xfrac}

\usepackage{enumerate} % Custom item numbers for enumerations

\usepackage[ruled]{algorithm2e} % Algorithms

\usepackage[framemethod=tikz]{mdframed} % Allows defining custom boxed/framed environments

\usepackage{listings} % File listings, with syntax highlighting
\lstset{
	basicstyle=\ttfamily, % Typeset listings in monospace font
}

\usepackage[ddmmyyyy]{datetime}

\geometry{
	paper=a4paper, % Paper size, change to letterpaper for US letter size
	top=3cm, % Top margin
	bottom=3cm, % Bottom margin
	left=2.5cm, % Left margin
	right=2.5cm, % Right margin
	headheight=25pt, % Header height
	footskip=1.5cm, % Space from the bottom margin to the baseline of the footer
	headsep=1cm, % Space from the top margin to the baseline of the header
	%showframe, % Uncomment to show how the type block is set on the page
}
\lhead{Analysis und Numerik\\Sommersemester 2025}
\chead{\bfseries{\vspace{0.5\baselineskip}Übungsblatt 4}}
\rhead{Lienkamp, 8128180\\Werner, 7987847}
\fancyheadoffset[R]{0cm}

\begin{document}
\setcounter{section}{4}
\subsection{Votieraufgabe}
Zeigen Sie
\begin{enumerate}[label=({\alph*})]
    \item Für $|q| < 1$ gilt $\lim\limits_{n\to\infty}q^n=0$.
    \item Für $q > 1$ gilt $\lim\limits_{n\to\infty}q^n = \infty$.
    \item Für $q\leq -1$ divergiert $q^n$ unbestimmt.
\end{enumerate}

\subsection{Multiple Choice}
Kreuzen Sie bei den folgenden Fragen jeweils die zutreende Antwort an. Korrekte Kreuze bringen +0.5 Punkte. Falsche Kreuze -0.5 Punkte. Sie bekommen auf diese Aufgabe mindestens 0 Punkte.\\
Seien $(a_n)n, (b_n)n \subset \mathbb{R}$ reelle Folgen.
\begin{enumerate}[label=({\alph*})]
    \item Die Reihe
    \[\sum\limits^\infty_{n=1}\frac{(-1)^{n+1}}{n}\]
    konvergiert absolut.
    \begin{flushright}
        wahr $\square\quad$ falsch $\boxtimes$
    \end{flushright}
    \textcolor{gray}{Die Folge ist zwar konvergent, jedoch nicht absolut konvergent, da $\sum\limits^\infty_{n=0}|a_n|$ nicht konvergiert.}
    \item Sei die Reihe $\sum\limits^\infty_{n=0}b_n$ konvergent und $a_n\leq b_n$ für alle $n\in\mathbb{N}$. Dann ist auch die Reihe $\sum\limits^\infty_{n=0}a_n$ konvergent.
    \begin{flushright}
        wahr $\square\quad$ falsch $\boxtimes$
    \end{flushright}
    \textcolor{gray}{$a_n$ ist nur nach oben beschränkt, deshalb ist die Konvergenz nicht gegeben.}
    \item Sei die Reihe $\sum\limits^\infty_{n=0}b_n$ konvergent und $|a_n|\leq |b_n|$ für alle $n\in\mathbb{N}$. Dann ist auch die Reihe $\sum\limits^\infty_{n=0}a_n$ konvergent.
    \begin{flushright}
        wahr $\boxtimes\quad$ falsch $\square$
    \end{flushright}
    \textcolor{gray}{Da durch den Betrag von $a_n$ gilt $0\leq |a_n|\leq b_n$ $a_n$ durch die x-Achse begrenzt ist, wissen wir, dass die Reihe konvergent ist.}
    \item Sei die Reihe $\sum\limits^\infty_{n=0}b_n$ konvergent und $|a_n|\leq b_n$ für alle $n\geq 100$. Dann sind die Reihen $\sum\limits^\infty_{n=0}a_n$ und $\sum\limits^\infty_{n=0}b_n$ absolut konvergent.
    \begin{flushright}
        wahr $\boxtimes\quad$ falsch $\square$
    \end{flushright}
    \textcolor{gray}{$\sum\limits^\infty_{n=0}a_n$ ist absolut konvergent, weil $\sum\limits^\infty_{n=0}b_n$ konvergiert und $|a_n|\leq b_n$ gilt. Konvergiert der Betrag einer Folge gilt sie als absolut konvergent. $b_n$ ist durch den Betrag von $a_n$ nach unten beschränkt und damit immer $\geq 0$ und somit auch absolut konvergent.}
\end{enumerate}
\subsection{Rechenaufgabe}
Berechnen Sie die folgenden Ausdrücke und tragen Sie diese in die Kästchen ein. Bitte geben Sie nur die Endergebnisse an. Korrekte Lösungen bringen +0.5 Punkte, falsche Lösungen 0 Punkte.\\\\
\fbox{\parbox{\linewidth}{
\[\sum\limits^\infty_{k=1}\left(-\frac{1}{3}\right)^k=-\frac{1}{4}\]
}}\vspace*{5mm}
\fbox{\parbox{\linewidth}{
\[\sum\limits^\infty_{k=5}\left(\frac{1}{2}\right)^{2k+1}=\frac{1}{6}\]
}}\vspace*{5mm}
\fbox{\parbox{\linewidth}{
\[\sum\limits^\infty_{k=1}\left(\frac{1}{2k+1}-\frac{1}{2k+2}\right)=0\]
}}\vspace*{5mm}
\fbox{\parbox{\linewidth}{
\[\sum\limits^\infty_{k=1}\left(-\frac{1}{3}\right)^k=-\frac{1}{4}\]
}}\vspace*{5mm}
\textit{Hinweis:} Vergewissern Sie sich zunächst, dass
\[\left(\frac{1}{2k+1}-\frac{1}{4k+2}-\frac{1}{4k+4}\right)=\frac{1}{2}\left(\frac{1}{2k+1}-\frac{1}{2k+2}\right).\]
\subsection{Schriftliche Aufgabe}
Sei $x>0$ fest. Zeigen Sie, dass die Folge
\[a_n=\left(1+\frac{x}{n}\right)^n\quad \text{ für } n\in\mathbb{N}\]
monoton steigend und beschränkt ist.\\
\textit{Hinweis:} Zeigen Sie zunächst, dass $\frac{a_{n+1}}{a_n}>1$ für alle $n\in\mathbb{N}$.
\end{document}
