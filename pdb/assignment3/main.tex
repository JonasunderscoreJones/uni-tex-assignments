\documentclass[a4paper]{article}
%\usepackage[singlespacing]{setspace}
\usepackage[onehalfspacing]{setspace}
%\usepackage[doublespacing]{setspace}
\usepackage{geometry} % Required for adjusting page dimensions and margins
\usepackage{amsmath,amsfonts,stmaryrd,amssymb,mathtools,dsfont} % Math packages
\usepackage{tabularx}
\usepackage{colortbl}
\usepackage{listings}
\usepackage{amsmath}
\usepackage{amssymb}
\usepackage{amsthm}
\usepackage{subcaption}
\usepackage{float}
\usepackage[table,xcdraw]{xcolor}
\usepackage{changepage,titlesec,fancyhdr} % For styling Header and Titles
\pagestyle{fancy}

\usepackage{enumerate} % Custom item numbers for enumerations

\usepackage[ruled]{algorithm2e} % Algorithms

\usepackage[framemethod=tikz]{mdframed} % Allows defining custom boxed/framed environments

\usepackage{listings} % File listings, with syntax highlighting
\lstset{
	basicstyle=\ttfamily, % Typeset listings in monospace font
}

\usepackage[ddmmyyyy]{datetime}


\geometry{
	paper=a4paper, % Paper size, change to letterpaper for US letter size
	top=2.5cm, % Top margin
	bottom=3cm, % Bottom margin
	left=2.5cm, % Left margin
	right=2.5cm, % Right margin
	headheight=25pt, % Header height
	footskip=1.5cm, % Space from the bottom margin to the baseline of the footer
	headsep=1cm, % Space from the top margin to the baseline of the header
	%showframe, % Uncomment to show how the type block is set on the page
}
\lhead{PDB\\Sommersemester 2024}
\chead{\bfseries{Übungsblatt 3}\\}
\rhead{7987847\\Jonas Werner}

\begin{document}
\section*{Aufgabe 1}
\subsection*{a)}

Das ER-Diagramm beschreibt den Tabelleneitrag einer Liga.\\
$Mannschaft = ({{Spieltermin}}, {Name})$\\
$dom(Spieltermin) = date$

\subsection*{b)}

Das ER-Diagramm beschreibt einen Personalausweis.\\
$Person = ({Name(Vorname, Nachname), Adresse(Straße, Nummer, Ort)}, {PersonID})$\\
$dom(Adresse) = char(30) \times int x\times char(30)$

\subsection*{c)}

Das ER-Diagramm beschreibt ein Szenario einer Arztpraxis wo Medikamente verschrieben werden.\\
$verschreibt = ((Ärzt*in, Medikament, Patient*in), (Datum))$\\
$grad(verschreibt) = 3$\\
$comp(verschreibt, (Ärzt*in, Medikament, Patient*in)) = $

\section*{2}
\subsection*{a)}
1. Es gibt keinee Unique ID: Zwei Zeilen können exakt gleich sein\\
2. Telefonnummer ist nicht einheitlich: Vorwalen, Schreibweisen, ...\\
3. Name ist nicht einheitlich: Namen pro Telefonnummer, Abkürzungen
\subsection*{b)}
1. Es kann eine ID Spalte eingeführt werden um jedem Eintrag eine eine eindeutige Zahl zuzuweisen.\\
2. Es kann eine Vorwahl-Spalte eingeführt werden und die Telefonnummer ohne Leerzeichen also $int$ zu speichern.\\
3. Name in Vor- und Nachname spalten und keine Abürzungen erlauben. Nur ein Name pro Eintrag.
\end{document}