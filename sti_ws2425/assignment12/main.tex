\documentclass[a4paper]{article}
%\usepackage[singlespacing]{setspace}
\usepackage[onehalfspacing]{setspace}
%\usepackage[doublespacing]{setspace}
\usepackage{geometry} % Required for adjusting page dimensions and margins
\usepackage{amsmath,amsfonts,stmaryrd,amssymb,mathtools,dsfont} % Math packages
\usepackage{tabularx}
\usepackage{colortbl}
\usepackage{listings}
\usepackage{amsmath}
\usepackage{amssymb}
\usepackage{amsthm}
\usepackage{subcaption}
\usepackage{float}
\usepackage[table,xcdraw]{xcolor}
\usepackage{tikz-qtree}
\usepackage{forest}
\usepackage{changepage,titlesec,fancyhdr} % For styling Header and Titles
\usepackage{amsmath}
\pagestyle{fancy}
\usepackage{diagbox}
\usepackage{xfrac}
\usepackage{pgfplots}
\pgfplotsset{compat=1.18}

\usepackage{enumerate} % Custom item numbers for enumerations
\usepackage{enumitem}

\usepackage[ruled]{algorithm2e} % Algorithms

\usepackage{hyperref}

\usepackage[framemethod=tikz]{mdframed} % Allows defining custom boxed/framed environments

\usepackage{listings} % File listings, with syntax highlighting
\lstset{
	basicstyle=\ttfamily, % Typeset listings in monospace font
}

\usepackage[ddmmyyyy]{datetime}


\geometry{
	paper=a4paper, % Paper size, change to letterpaper for US letter size
	top=2.5cm, % Top margin
	bottom=3cm, % Bottom margin
	left=2.5cm, % Left margin
	right=2.5cm, % Right margin
	headheight=25pt, % Header height
	footskip=1.5cm, % Space from the bottom margin to the baseline of the footer
	headsep=1cm, % Space from the top margin to the baseline of the header
	%showframe, % Uncomment to show how the type block is set on the page
}
\lhead{Stochastik für die Informatik\\Wintersemester 2024/2025}
\chead{\bfseries{Übungsblatt 12}\\}
\rhead{Lienkamp, 8128180\\Werner, 7987847}
\begin{document}
\setcounter{section}{12}
\subsection{$\mathcal{X}^2$-Verteilung}
Seien $\mathcal{X}_1 \sim \mathcal{N}(0,1)$ und $\mathcal{X}_2 \sim \mathcal{N}(0,2.25)$ unabhängige Zufallsvariablen. Geben Sie den numerischen Wert der Wahrscheinlichkeit $\mathbb{P}(4X^2_2 \leq 4.5-9X^2_1)$ an.\\
\textit{Hinweis}: in Excel/LibreOffice (englische Version) gibt die Funktion \texttt{CHISQ.DIST(x,n,1)} die Wahrscheinlichkeit $\mathbf{P}(X \leq x)$ für eine $\mathcal{X}^2$-verteilte Zufallsvariable $X$ mit Parameter $n$.\\\\
Gegeben:
\begin{itemize}
    \item Ungleichung: $4X^2_2 \leq 4.5-9X^2_1$
    \item Die \textbf{unabhängigen} standardnormalverteilten Zufallsvariablen $\mathcal{X}_1 \sim \mathcal{N}(0,1)$ und $\mathcal{X}_2 \sim \mathcal{N}(0,2.25)$
\end{itemize}]
Gesucht:
\begin{itemize}
    \item $\mathbb{P}(4X^2_2 \leq 4.5-9X^2_1)$
    \item Zunächst:
    \begin{itemize}
        \item Quantil $x$
    \end{itemize}
\end{itemize}
In unserer Ungleichung $4X^2_2 \leq 4.5-9X^2_1$ können wir folgendes \textit{Substituieren}:\\
\[\mathcal{X}_2 \text{ zu } 1.5Z \text{ mit } Z \sim \mathcal{N}(0,1)\]
Somit wird $4X^2_2 \leq 4.5-9X^2_1$ zu $4(1.5Z)^2 \leq 4.5-9X^2_1$\\
\hspace*{3.57cm}$= 4 \cdot 2.25Z^2 + 9X_1^2 \leq 4.5$\\
\hspace*{3.57cm}$= 9Z^2 + 9X_1^2 \leq 4.5 \qquad\qquad\qquad\qquad \vert :9$\\
\hspace*{3.57cm}$= Z^2 + X_1^2 \leq 0.5$\\\\
Sowohl $\mathcal{X}_1$ als auch $\mathbb{Z}$ sind standardnormalverteilt und unabhängig. Somit haben wir 2 Freiheitsgrade und die Summe der \textit{Chi-Quadrat Verteilung} ist gegeben durch:
\[\mathcal{X}_1^2 + \mathbb{Z}^2 \sim \mathcal{X}^2(2)\]
Somit ist die Ungleichung:\\
\[\mathcal{X}_1^2 + \mathbb{Z}^2 \leq 0.5 \Rightarrow \mathcal{X}^2 \leq 0.5\]
$\Rightarrow \mathbb{P}(\mathcal{X}^2) \leq 0.5$\\\\
Damit haben wir die folgenden Werte für unsere LibreOffice-Funktion gefunden:\\
\begin{itemize}[label=\(\rightarrow\)]
    \item Quantil $x = 0.5$
    \item Freiheitsgrade $n = 2$
    \item Kumulativität $cum = 1$
\end{itemize}
Eingesetzt in die LibreOffice-Formel \texttt{CHISQ.DIST(x, n, cum)} ergibt sich dann:\\\\
\texttt{CHISQ.DIST(0.5, 2, 1)} $\approx 0.221$.
\subsection{Konfidenzbereiche der Binomialverteilung}
Der Hersteller eines neuen Medikaments behauptet, die Heilungswahrscheinlichkeit des Medikaments betrage mindestens $p_\ast = 0.9$. Um diese Behauptung zu prüfen, wird dieses Medikament in einer Studie mit 300 Personen getestet. Sei $X$ die Anzahl der Behandlungserfolge.\\\\
(a) Welche Konfidenzintervalle sind hier geeignet, um eine solche Bahuaptung zu stützen oder zu zerlegen?\\\\
Gesucht ist ein unteres Konfidenzintervall für den Anteil \( p \) konstruieren, um die Behauptung, dass \( p \geq 0.9 \) ist zu überprüfen.

Aus der Vorlesung wissen wir, dass für ein einseitiges unteres Konfidenzintervall bei einem Anteilswert die Formel gilt:
\[
\text{Untere Grenze} = \hat{p} - z_{\alpha} \cdot \sqrt{\frac{\hat{p}(1 - \hat{p})}{n}}
\]
\begin{itemize}
    \item \(\hat{p}\): Stichprobenanteil (z.~B. aus Teil (b): \( \hat{p} = 0.8667 \))
    \item \( z_{\alpha} \): Kritischer Wert der Standardnormalverteilung für das Signifikanzniveau \( \alpha \). 
      \begin{itemize}
          \item Bei \( \alpha = 0.04 \) (96\%-KI): \( z_{0.04} \approx -1.75 \).
          \item Den Wert können wir aus der Standardnormalverteilungstabelle ablesen: Fläche \( 1 - \alpha = 0.96 \).
      \end{itemize}
    \item \( n \): Stichprobenumfang (\( n = 300 \)).
\end{itemize}
Nun müssen wir das Ergebnis noch interpretieren:
\begin{itemize}
    \item Liegt die untere Grenze \textbf{über} \( 0.9 \), unterstützt dies die Behauptung \( p \geq 0.9 \).
    \item Liegt sie \textbf{unter} \( 0.9 \), ist die Behauptung statistisch fragwürdig.
\end{itemize}
(b) Angenommen, die Studie lieferte 260 Behandlungserfolge. Widerspricht das Ergebnis der Behauptung des Herstellers? Nutzen Sie das Niveau $\alpha = 0.04$ zur Berechnung des Konfidenzintervalls.\\\\
Wir stelle die Hypothesen auf:
\[
H_0: p \geq 0.9 \quad \text{(Nullhypothese)} \qquad \text{und} \qquad H_1: p < 0.9 \quad \text{(Alternativhypothese)}
\]

\subsubsection*{Schritt 2: Teststatistik berechnen}
Die Teststatistik (z-Wert) kann mit der folgenden Formel aus der Vorlesung berechnet werden:
\[
z = \frac{\hat{p} - p_0}{\sqrt{\frac{p_0(1 - p_0)}{n}}}
\]
\begin{itemize}
    \item \( p_0 = 0.9 \): der behauptete Anteil von der Hypothese \( H_0 \).
    \item \( \hat{p} = \frac{260}{300} \approx 0.8667 \): der beobachtete Anteil.
    \item Die Werte werden eingesetzt:
    \[
    z = \frac{0.8667 - 0.9}{\sqrt{\frac{0.9 \cdot 0.1}{300}}} = \frac{-0.0333}{\sqrt{0.0003}} \approx \frac{-0.0333}{0.0173} \approx -1.92
    \]
\end{itemize}
Nun bestimmen wir die wichtigen bzw. kritischen Werte:
\begin{itemize}
    \item Das Signifikanzniveau: \( \alpha = 0.04 \).
    \item Ein Einseitiger Test: Kritischer Wert \( z_{0.04} \approx -1.75 \) (aus der Tabelle).
\end{itemize}
Nun die Testentscheidung:
\[
\text{Vergleich: } z_{\text{berechnet}} = -1.92 \quad vs. \quad z_{\text{kritisch}} = -1.75
\]
\begin{itemize}
    \item Weil folgendes gilt: \( -1.92 < -1.75 \), liegt die Teststatistik im Ablehnungsbereich.
    \item Die Schlussfolgerung: Die Daten widersprechen der Behauptung \( p \geq 0.9 \).
\end{itemize}
Wir müssen nun noch die Vorraussetzungen überprüfen:\\
Die Normalapproximation gilt, weil:
\[
n \cdot p_0 = 300 \cdot 0.9 = 270 \geq 10 \quad \text{und} \quad n \cdot (1 - p_0) = 300 \cdot 0.1 = 30 \geq 10
\]

\subsection{Konfidenzbereiche der Normalverteilung}
Die Anzahl der Bäckereien in einzigen Bezirken von West-Berlin in 1988 wurde wie folgt gegeben:\\\\
\begin{tabular}{c|c|c|c|c|c|c|c|c}
    Bezirk & Kreuzbg. & Wedding & Tiergarten & Neukölln & Spandau & Charl. & Temp. & Reinick. \\
    \hline
    Anzahl & 78 & 83 & 65 & 98 & 82 & 65 & 61 & 68 \\
\end{tabular}\\\\
(a) Zuerst werden diese als normalverteilt angenommen mit unbekanntem Erwartungswert und unbekannter Varianz. Bestimmen Sie ein Konfidenzintervall für $\mu$ zum Niveau $\alpha = 0.04$. \href{https://planetcalc.com/5017/}{Berechnung 0,98 Quantil}\\\\
\(\bar{\mu}= \frac{600}{8}=75\qquad \alpha = 0,04 \qquad a = 0,02\)\\
\(\sigma^2=\frac{1}{8-1}\sum\limits^8_{i=1}(x_i-\bar{\mu})^2 = \frac{(78-75)^2+(83-75)^2+(65-75)^2+(98-75)^2+(82-75)^2+(65-75)^2+(61-75)^2+(68-75)^2}{8-1}\)\\
\(= \frac{9+64+100+529+49+100+196+49}{7}=\frac{1096}{7}\)\\\\
\(J=\left[\sigma^2\frac{n-1}{\mathcal{X}^2_{n-1,1-\alpha/2}}, \sigma^2\frac{n-1}{\mathcal{X}^2_{n-1,\alpha/2}}\right]\)\\
\hspace*{0.3cm}\(\Rightarrow \left[75-\sqrt{\frac{\frac{1096}{7}}{8}}\bar{\mu}_{7,\, 0,98}; 75 + \sqrt{\frac{\frac{1096}{7}}{8}}\bar{\mu}_{7,\, 0,98}\right]= \left[75-\sqrt{\frac{\frac{1096}{7}}{8}}\cdot 2,5168; 75 + \sqrt{\frac{\frac{1096}{7}}{8}}\cdot 2,5168\right] = \left[63,87;86,13\right]\)\\\\
(b) Anschließend werden diese als normalverteilt angenommen mit bekanntem Erwartungswert 77. Bestimmen Sie ein Konfidenzintervall für $\sigma^2$ zum Niveau $\alpha = 0.02$.\\\\
\(\bar{\mu}=77 \qquad \alpha = 0,02\)\\
\(\sigma^2=\frac{1}{8-1}\sum\limits^8_{i=1}(x_i-\bar{\mu})^2 =\frac{(78-77)^2+(83-77)^2+(65-77)^2+(98-77)^2+(82-77)^2+(65-77)^2+(61-77)^2+(68-77)^2}{8-1}\)\\
\(=\frac{1+36+144+441+25+144+256+81}{7}=\frac{1128}{7}\)\\
\(J=\left[\sigma^2\frac{n-1}{\mathcal{X}^2_{n-1,1-\alpha/2}}, \sigma^2\frac{n-1}{\mathcal{X}^2_{n-1,\alpha/2}}\right]\)\\
\hspace*{0.3cm}\(\Rightarrow \left[\frac{1128}{7}\cdot \frac{7}{\bar{\mu}^2_{7,\,0,99}}; \frac{1128}{7}\cdot \frac{7}{\bar{\mu}^2_{7,\,0,01}}\right]=\left[\frac{1128}{18,475}; \frac{1128}{1,238}\right]=[61,055;911,147]\)\\\\
\subsection{\textit{t}-Test}
Aus einer Herde von Milchkühen wurden 14 Kühe ausgewählt und die Menge ihrer abgegebenen Milch in Kilogramm in einer Woche aufgezeichnet. Heraus kamen
\begin{center} \(169.6\qquad142.0\qquad103.3\qquad111.6\qquad123.4\qquad143.5\qquad155.1\)\\
\(101.7\qquad170.7\qquad113.2\qquad130.9\qquad146.1\qquad169.3\qquad155.5\)
\end{center}
Welche sinnvollen Annahmen sollten Sie an die Verteilung stellen? Was lässt sich über die Behauptung, die Milchproduktion liegt über 120kg pro Woche, aussagen?\\\\
Wir nehmen an, dass die Daten normalverteilt ist. Die Kühe sind also unabhängig voneinander.\\
Wir nutzen:
Nullhypothese (\( H_0 \)): Die Milchproduktion ist im Schnitt 120 kg (\( \mu = 120 \)).\\
Alternativhypothese (\( H_1 \)): Die Milchproduktion liegt über 120 kg (\( \mu > 120 \)).\\\\
Mittelwert: \( \bar{x} = \frac{1935.9}{14} \approx 138.28 \) kg
Standardabweichung: \( s \approx 25.67 \) kg\\\\
\(t = \frac{138.28 - 120}{25.67 / \sqrt{14}} \approx 2.66\)\\\\
Kritischer \textit{t}-Wert (bei \( \alpha = 0.05 \)): 1.771\\
Da \( 2.66 > 1.771 \), lehnen wir \( H_0 \) ab.\\\\
Die Milchproduktion liegt im Schnitt deutlich über 120 kg pro Woche (\( t(13) = 2.66, p < 0.05 \)).\\\\
Die Kühe geben im Schnitt mehr als 120 kg Milch pro Woche.\\\\
\textbf{Hinweise zur Bearbeitung der Aufgaben:}
\begin{itemize}
    \item Die Hausaufgabenblätter werden Freitags auf Moodle veröffentlicht und enthalten Hausaufgaben, die in der darauf folgenden Woche entweder \textbf{vor der Vorlesung am Freitag um 12:00} Uhr in Hörsaal V abzugeben sind oder \textbf{vor Freitag 12:00 Uhr} in das Schließfach Ihres Tutors (Robert-Mayer-Straße 6-8, 3. Stock) eingeworfen werden müssen.
    \item Die Hausaufgaben werden anschließend in den Tutorien der nächsten Woche besprochen.
\end{itemize}

\end{document}
