\documentclass[a4paper]{article}
%\usepackage[singlespacing]{setspace}
\usepackage[onehalfspacing]{setspace}
%\usepackage[doublespacing]{setspace}
\usepackage{geometry} % Required for adjusting page dimensions and margins
\usepackage{amsmath,amsfonts,stmaryrd,amssymb,mathtools,dsfont} % Math packages
\usepackage{tabularx}
\usepackage{colortbl}
\usepackage{listings}
\usepackage{amsmath}
\usepackage{amssymb}
\usepackage{amsthm}
\usepackage{subcaption}
\usepackage{float}
\usepackage[table,xcdraw]{xcolor}
\usepackage{tikz-qtree}
\usepackage{forest}
\usepackage{changepage,titlesec,fancyhdr} % For styling Header and Titles
\usepackage{amsmath}
\pagestyle{fancy}
\usepackage{diagbox}
\usepackage{xfrac}

\usepackage{enumerate} % Custom item numbers for enumerations

\usepackage[ruled]{algorithm2e} % Algorithms

\usepackage[framemethod=tikz]{mdframed} % Allows defining custom boxed/framed environments

\usepackage{listings} % File listings, with syntax highlighting
\lstset{
	basicstyle=\ttfamily, % Typeset listings in monospace font
}

\usepackage[ddmmyyyy]{datetime}


\geometry{
	paper=a4paper, % Paper size, change to letterpaper for US letter size
	top=2.5cm, % Top margin
	bottom=3cm, % Bottom margin
	left=2.5cm, % Left margin
	right=2.5cm, % Right margin
	headheight=25pt, % Header height
	footskip=1.5cm, % Space from the bottom margin to the baseline of the footer
	headsep=1cm, % Space from the top margin to the baseline of the header
	%showframe, % Uncomment to show how the type block is set on the page
}
\lhead{Stochastik für die Informatik\\Wintersemester 2024/2025}
\chead{\bfseries{Übungsblatt 3}\\}
\rhead{Lienkamp, 8128180\\Werner, 7987847}

\begin{document}
\setcounter{section}{4}
\subsection{}
Es seien $X$ und $Y$ zwei Zufallsvariablen mit gemeinsamer Verteilung $p_{X,Y}$ gegeben durch\\
\begin{table}[ht]
\centering
\begin{tabular}{c|c c c c}
\diagbox{x}{y} & 1 & 2 & 3 & 4 \\ \hline
1 & $\sfrac{2}{8}$ & 0 & $\sfrac{1}{4}$ & $\sfrac{1}{8}$ \\
4 & 0 & $\sfrac{1}{8}$ & $\sfrac{1}{4}$ & 0 \\
\end{tabular}
\end{table}\\
Berechnen Sie die Wahrscheinlichkeit der folgenden Ereignisse
\begin{center}
  \hfill (a) $X = Y$, \hfill (b) $X \leq Y$, \hfill (c) $X < Y$, \hfill (d) $Y + X$ ist gerade. \hfill
\end{center}
Bestimmen und skizzieren Sie die Verteilungsfunktion der Zufallsvariablen $Z := X \cdot Y$.
\subsubsection*{(a) $X = Y$}
$\mathbb{X = Y} = \mathbb{P}(X = Y = 1) + \mathbb{P}(X = Y = 2) = \frac{2}{8} + \frac{1}{8} = \frac{3}{8}$
\subsubsection*{(b) $X \leq Y$}
$\mathbb{P}(X \leq Y) = \mathbb{P}(X = Y = 1) + \mathbb{P}((X = 1) \cap (Y = 1)) + \mathbb{P}(X = Y = 2) + \mathbb{P}((X = 1) \cap (Y = 1)) + \\
\mathbb{P}((X = 1) \cap (Y = 3)) + \mathbb{P}((X = 2) \cap (Y = 3)) + \mathbb{P}((X = 1) \cap (Y = 4)) + \mathbb{P}((X = 2) \cap (Y = 3)) + \\
\mathbb{P}((X = 2) \cap (Y = 4)) = \frac{2}{8} + 0 + \frac{1}{8} + \frac{1}{4} + \frac{1}{4} + \frac{1}{8} + 0 = 1$
\subsubsection*{(c) $X < Y$}
$\mathbb{P}(X < Y) = \mathbb{P}(X \leq Y) - \mathbb{X = Y = 1} - \mathbb{P}(X = Y = 2) = 1 - \frac{2}{8} - \frac{1}{8} = \frac{5}{8}$
\subsubsection*{(d) $Y + X$}
$\mathbb{P}((X + Y) \mod 2 = 0) = \mathbb{P}(X = Y = 1) + \mathbb{P}(X = Y = 2) + \mathbb{P}((X = 1) \cap (Y = 3)) +\\
\mathbb{P}((X = 2) \cap (Y = 4)) = \frac{2}{8} + \frac{1}{8} + \frac{1}{4} + 0 = \frac{5}{8}$
\subsection{}
Ein Flugzeug auf der Langstrecke von Berlin nach New York habe 298 Sitzplätze. Typischerweise erscheinen von den Personen, die einen Flug auf dieser Strecke buchen, etwa 3\% nicht rechtzeitig.\\
Angenommen alle Plätze seien gebucht.\\\\
(a) Wie groß ist die Wahrscheinlichkeit, dass höchstens 2, 5 oder 10 Plätze nicht besetzt sind?\\\\
\(\sum\limits^x_{k=0} (0,03)^k\cdot (0,97)^{298-k} \cdot \binom{298}{k}\\
\text{für }x=2 :\sum\limits^2_{k=0} (0,03)^k\cdot (0,97)^{298-k} \cdot \binom{298}{k} \approx 0,006005 \approx 0,6\%\\\\
\text{für }x=5 :\sum\limits^5_{k=0} (0,03)^k\cdot (0,97)^{298-k} \cdot \binom{298}{k} \approx 0,115669 \approx 11,6\%\\\\
\text{für }x=10 :\sum\limits^{10}_{k=0} (0,03)^k\cdot (0,97)^{298-k} \cdot \binom{298}{k} \approx 0,715004 \approx 71,5\%\)
\clearpage 
\noindent(b) Um mit weniger leeren Plätzen zu fliegen, möchte die Fluggesellschaft mehr Buchungen als zur Verfügung stehende Plätze zulassen. Wieviele zusätzliche Tickets kann sie verkaufen, sodass mit einer Wahrscheinlichkeit von mindestens 95\% alle erscheinenden Passagiere mitfliegen können?\\
$Hinweis$: Schauen sie sich die Verteilungsfunktion für verschiedene Anzahlen von Buchungen an.\\\\
Aus Aufgabe 2.a  haben wir zwar die Wahrscheinlichkeiten für 2, 5 und 10 fehlende Passagiere gegeben, hier benötigen wir jedoch nicht die Wahrscheinlichkeit, dass höchstens $x$ Passagiere bei 298 Tickets fehlen, sondern, dass genau die Anzahl an Leuten fehlt, für die zu viele Tickets verkauft wurden.\\
Wir wissen, dass bei 298 verkauften Tickets die Wahrscheinlichkeiten zwischen 2 und 5 fehlenden Passagieren für unsere Fragestellung gut in Betracht kommen könnten. Da wir die höchste Anzahl an Tickets suchen, die verkauft werden können und mit einer Wahrscheinlichkeit von mindestens 95\% alle Passagiere, die pünktlich erschienen sind mitfliegen sollen, beginnen wir mit 4 Tickets zu viel.
Dafür berechnen wir:\\\\
Für 4 Tickets zu viel ziehen wir alle Möglichkeiten ab, die nicht eintreten dürfen, also, dass höchstens 3 Leute zu spät kommen:\\
\(1-\sum\limits^3_{k=0}(0,03)^k\cdot (0,97)^{301-k} \cdot \binom{301}{k} \approx 0.980535 \geq 95\%\)\\\\
Die Wahrscheinlichkeit bei 4 Tickets zu viel liegt bei über 95\%, wir berechnen aber trotzdem noch die Wahrscheinlichkeit für 5 Tickets zu viel, da wir die maximale Anzahl an mehrverkäuflichen Tickets suchen:
\(1 -\sum\limits^4_{k=0}(0,03)^k\cdot (0,97)^{302-k} \cdot \binom{302}{k} \approx 0.949502 \leq 95\%\)\\\\
Die Fluggesellschaft kann vier weitere Tickets verkaufen und mit einer Wahrscheinlichkeit von über 95\% können alle Passagiere, die erscheinen auch mitfliegen.
\subsection{}
Die Anzahl der Studierenden, die eine Sprechstunde für Stochastik für Informatiker besuchen, ist poissonverteilt mit Parameter $\lambda > 0$. Diese Studierende studieren jeweils mit Wahrscheinlichkeit $p$ unabhängig voneinander Informatik, wobei $p \in (0, 1)$.\\\\
(a) Bestimmen Sie die Wahrscheinlichkeit, dass genau $k$ Informatikstudierende zur Sprechstunde kommen, falls insgesamt $n$ Studierende zur Sprechstunde kommen, für $k, n \in \mathbb{N}_0$.\\\\
X ist die Anzahl an Studierenden in der Sprechstunde\\
Y ist die Anzahl der Studierenden, welche die Sprechstunde besuchen, Poisson-verteilt\\
Aus der Vorlesung ist bekannt, dass $\mathbb{P}(Y = n) = \frac{\lambda^n}{n!}\cdot e^{-\lambda}$\\
Analog gilt $\mathbb{P}(X = k) = \frac{\lambda^k}{k!}\cdot e^{-\lambda}$\\
Außerdem gilt aus der Vorlesung: $\mathbb{P}(X = k\ |\ Y = n) = \binom{n}{p} \cdot p^\lambda \cdot (1 - p)^{n - k}$\\
$\mathbb{P}(X = k\ |\ Y = n) = \frac{\mathbb{P}((X = k) \cap (Y = n))}{\mathbb{P}(Y = n)} = \frac{ \mathbb{P}(Y = n) \cdot \mathbb{P}(X = k\ |\ Y = n)}{\mathbb{P}(Y = n)} = \frac{\sfrac{\lambda^n}{n!}\cdot e^{-\lambda} \cdot \binom{n}{p} \cdot p^\lambda \cdot (1 - p)^{n - k}}{\sfrac{\lambda^n}{n!} \cdot e^{-\lambda}} = \binom{n}{p} \cdot p^\lambda \cdot (1 - p)^{n - k}$\\
Daraus folgt dann:\\
$\mathbb{P}(X = k\ |\ Y = n) =
\begin{cases}
    \text{$\binom{n}{p} \cdot p^\lambda \cdot (1 - p)^{n - k}$} & \text{, } k \leq n \\
    \text{0} & \text{, } k > n
\end{cases}$\\
Das liegt daran, dass das Ergebnis nur Sinn ergibt, wenn $k \leq n$ ist, weil nicht mehr Studenten Die Sprechstunde besuchen, als in der Sprechstunde sein können. Die Wahrscheinlichkeit für  $k > n$ ist daher 0.
\\\\
(b) Wie ist die Anzahl der Informatikstudierenden bei der Sprechstunde verteilt? Geben Sie die Verteilung dieser Zufallsvariable explizit an.
\subsection{Zipf-Verteilung}
In einem Land gibt es abzählbar unendlich viele Städte die durchnummeriert sind mit den Zahlen {1,2,3...}. Die Wahrscheinlichkeit, dass ein zufällig ausgewählter Bewohner des Landes in Stadt $k$ lebt, sei gegeben durch $\frac{k-2}{\sfrac{\pi^2}{6}}$.\\\\
(a) Berechnen Sie die Wahrscheinlichkeit, dass ein zufällig ausgewählter Bewohner in einer der Städte {5,6,7,8,...} lebt.\\
X ist die Stadt in der der zufällig ausgewählte Bewohner lebt.
$\mathbb{P}(X \geq 5) = 1 - \mathbb{P}(X \leq 4)$\\
$\mathbb{P}(X \leq 4) = \sum^4_{k=1} \frac{k^{-2}}{\sfrac{\pi^2}{6}} = \frac{205}{24 \cdot \pi^2}$\\
$\Rightarrow{} \mathbb{P}(X \geq 5) = 1 - \mathbb{P}(X \leq 4) = 1 - \frac{205}{24 \cdot \pi^2} \approx 0.1345$\\
Die Wahrscheinlichkeit, dass ein zufällig ausgewählter Bewohner in einer der Städte 5,6,7,8,... lebt, liegt bei 13.45\%.\\\\
(b) Berechnen Sie die Wahrscheinlichkeit, dass zwei zufällig und unabhängig voneinander ausgewählte Bewohner in der selben Stadt wohnen.\\\\
$Hinweis$: Sie können die Gleichungen $\sum^{\infty}_{k=1}k^{-2} = \sfrac{\pi^2}{6} $ und $ \sum^{\infty}_{k=1} k^{-4} = \sfrac{\pi^4}{90}$ verwenden.\\
Wir definieren X und Y als die Städte in denen die beiden zufällog ausgewählten Bewohner leben.\\\\
$\mathbb{P}(X = Y) = \mathbb{P}(X = 1, Y = 1) + \mathbb{P}(X = 2, Y =2) + \dots = \sum^\infty_{k=1} \mathbb{P}(X = Y = k) = \sum^\infty_{k=1} \mathbb{P}(X = k) \cdot \mathbb{P}(Y = k) = \sum^\infty_{k=1} \left(\frac{k^{-2}}{\sfrac{\pi^2}{6}}\right)^2 = \sum^\infty_{k=1} \left(\frac{6}{\pi^2 k^2}\right)^2 = \sum^\infty_{k=1} \frac{36}{\pi^4k^4} = \frac{36}{\pi^4} \cdot \sum^\infty_{k=1} k^{-4} = \frac{36}{\pi^4} \cot \frac{\pi^4}{90} = \frac{36}{90} = \frac{2}{5}$
\end{document}
